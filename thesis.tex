\documentclass[11pt, twoside, ngerman, openright]{report}

\usepackage[utf8]{inputenc}
\usepackage{graphicx}
\usepackage[top=25mm, bottom=25mm, bindingoffset=6mm, includeheadfoot]{geometry}
\usepackage{fancyhdr}
\usepackage{emptypage}
\usepackage[style=alphabetic]{biblatex}
\usepackage[ngerman]{babel}
\usepackage[acronym]{glossaries}
\usepackage[onehalfspacing]{setspace}
\usepackage{titlesec}
\usepackage{hyperref}


\addbibresource{chapters/my_bib.bib}

\graphicspath{{images/}}

\makeglossaries
\loadglsentries[main]{chapters/glossaries}


\pagestyle{fancy}
\renewcommand{\chaptermark}[1]{\markboth{\thechapter.\ #1}{}}
\renewcommand{\sectionmark}[1]{\markright{\thesection.\ #1}{}}
\fancyhead{}
\fancyhead[RE]{\leftmark}
\fancyhead[LO]{\rightmark}
\fancyfoot{}
\fancyfoot[RE]{\footnotesize{Institut für Technik der Informationsverarbeitung\\
			   Karlsruhe Institut für Technologie}}
\fancyfoot[LO]{\footnotesize{\mytitle}}
\fancyfoot[LE,RO]{\thepage}
\renewcommand{\headrulewidth}{0.4pt}
\renewcommand{\footrulewidth}{0.4pt}
\fancypagestyle{plain}{
	\fancyhf{} % clear all header and footer fields
	\fancyfoot[LE,RO]{\thepage}
	\fancyfoot[LO]{\footnotesize{\mytitle}}
	\renewcommand{\headrulewidth}{0pt}
	\renewcommand{\footrulewidth}{0.4pt}}


\titleformat{\chapter}[display]
  {\Huge\bfseries}
  {}
  {0pt}
  {\thechapter.\ }

\titleformat{name=\chapter,numberless}[display]
  {\Huge\bfseries}
  {}
  {0pt}
  {}


\newcommand{\submissiondate}{23. Dezember 2018}
\newcommand{\mytitle}{Titel der Arbeit}



\begin{document}

\pagestyle{fancy}
\pagenumbering{Roman}

\input{chapters/titlepage}



\chapter*{Abstract}

Abstract \gls{itiv} goes \gls{t} here \cite{einstein}. test mit Ö, ä und ß ...
%\addcontentsline{toc}{chapter}{Abstract}

\input{chapters/declaration}
%\addcontentsline{toc}{chapter}{Eidesstattliche Erklärung}

\tableofcontents

\cleardoublepage
\pagenumbering{arabic}


% ===========================
\chapter{Einleitung}
\label{einleitung}
% ===========================


% ===========================
\section{Problemstellung und Motivation}
\label{einleitung_problemstellung}
% ===========================

Hochautomatisierte Fahrerassistenzsysteme werden zunehmend komplexer. Herkömmliche Testmethoden sind durch die Vielzahl an möglichen Szenarien nicht mehr praktisch testbar. Heutzutage wird schon vieles in Simulation getestet. Dabei git es aktuell noch Probleme...


% ===========================
\section{Zielsetzung}
\label{einleitung_zielsetzung}
% ===========================

Genau hier setzt diese Arbeit an.. 

Aufbau der Arbeit ist   




% ===========================
\chapter{Grundlagen}
\label{grundlagen}
% ===========================

Das Ziel dieses Kapitels ist es ... in italic


% ===========================
\section{Hochautomatisierte Fahren}
\label{grundlagen_fahren}
% ===========================


% ===========================
\subsection{Entwicklung hochautomatisierter Fahrfunktionen}
\label{grundlagen_fahren_entwicklung}
% ===========================

Lorem ipsum dolor sit amet, consetetur sadipscing elitr, sed diam nonumy eirmod tempor invidunt ut labore et dolore magna aliquyam erat, sed diam voluptua. At vero eos et accusam et justo duo dolores et ea rebum. Stet clita kasd gubergren, no sea takimata sanctus est Lorem ipsum dolor sit amet. Lorem ipsum dolor sit amet, consetetur sadipscing elitr, sed diam nonumy eirmod tempor invidunt ut labore et dolore magna aliquyam erat, sed diam voluptua. At vero eos et accusam et justo duo dolores et ea rebum. Stet clita kasd gubergren, no sea takimata sanctus est Lorem ipsum dolor sit amet. Lorem ipsum dolor sit amet, consetetur sadipscing elitr, sed diam nonumy eirmod tempor invidunt ut labore et dolore magna aliquyam erat, sed diam voluptua. At vero eos et accusam et justo duo dolores et ea rebum. Stet clita kasd gubergren, no sea takimata sanctus est Lorem ipsum dolor sit amet. 

Lorem ipsum dolor sit amet, consetetur sadipscing elitr, sed diam nonumy eirmod tempor invidunt ut labore et dolore magna aliquyam erat, sed diam voluptua. At vero eos et accusam et justo duo dolores et ea rebum. Stet clita kasd gubergren, no sea takimata sanctus est Lorem ipsum dolor sit amet. 


% ===========================
\subsection{Simulation in der Entwicklung}
\label{grundlagen_fahren_simulation}
% ===========================

Lorem ipsum dolor sit amet, consetetur sadipscing elitr, sed diam nonumy eirmod tempor invidunt ut labore et dolore magna aliquyam erat, sed diam voluptua. At vero eos et accusam et justo duo dolores et ea rebum. Stet clita kasd gubergren, no sea takimata sanctus est Lorem ipsum dolor sit wie in Graph \ref{fig:x cubed graph} beschrieben.Lorem ipsum dolor sit amet, consetetur sadipscing elitr, sed diam nonumy eirmod tempor invidunt ut labore et dolore magna aliquyam erat, sed diam voluptua. At vero eos et accusam et justo duo dolores et ea rebum. Stet clita kasd gubergren, no sea takimata sanctus est Lorem ipsum dolor sit wie in Graph \ref{fig:x cubed graph} beschrieben.

Lorem ipsum dolor sit amet, consetetur sadipscing elitr, sed diam nonumy eirmod tempor invidunt ut labore et dolore magna aliquyam erat, sed diam voluptua. At vero eos et accusam et justo duo dolores et ea rebum. Stet clita kasd gubergren, no sea takimata sanctus est Lorem ipsum dolor sit wie in Graph \ref{fig:x cubed graph} beschrieben.


\begin{figure}[h]
\centering
\includegraphics[scale=0.5]{graph_a}
\caption{An example graph}
\label{fig:x cubed graph}
\end{figure}

Duis autem vel eum iriure dolor in hendrerit in vulputate velit esse molestie consequat, vel illum dolore eu feugiat nulla facilisis at vero eros et accumsan et iusto odio dignissim qui blandit praesent luptatum zzril delenit augue duis dolore te feugait nulla facilisi.   


% ===========================
\section{Neuronale Netze}
\label{grundlagen_nn}
% ===========================

Duis autem vel eum iriure dolor in hendrerit in vulputate velit esse molestie consequat, vel illum dolore eu feugiat nulla facilisis.


% ===========================
\subsection{Maschinelles Lernen}
\label{grundlagen_nn_ml}
% ===========================

Duis autem vel eum iriure dolor in hendrerit in vulputate velit esse molestie consequat, vel illum dolore eu feugiat nulla facilisis at vero eros et accumsan et iusto odio dignissim qui blandit praesent luptatum zzril delenit augue duis dolore te feugait nulla facilisi. 


% ===========================
\subsection{Convolutional Neural Network}
\label{grundlagen_nn_cnn}
% ===========================


\begin{table}[h]
\centering
\begin{tabular}{l | l | l}
A & B & C \\
\hline
1 & 2 & 3 \\
4 & 5 & 6
\end{tabular}
\caption{very basic table}
\label{tab:abc}
\end{table}


% ===========================
\subsection{Recurrent Neural Network}
\label{grundlagen_nn_rnn}
% ===========================

Lorem ipsum dolor sit amet, consectetuer adipiscing elit, sed diam nonummy nibh euismod tincidunt ut laoreet dolore magna aliquam erat volutpat \cite{latexcompanion}. 




% ===========================
\chapter{Konzept}
\label{konzept}
% ===========================

Wie in Abschnitt \ref{grundlagen_fahren} beschrieben, stellt Sicherung von hochautomatisierten \ac{FAS} die Automobilindustrie vor große Herausforderungen. Die Menge der bekannten Fahrszenarien ist nur eine Teilmenge aller Szenarien, die zukünftige \ac{FAS} abdecken müssen. Diese Beziehung ist schematisch in Abbildung \ref{fig_teilmenge_fahrszenarien} dargestellt. Die Folge ist eine steigende Anzahl benötigter Testkilometer, die in Zukunft mit ökonomischem Aufwand nicht mehr umsetzbar sein wird. Es müssen neue Methoden gefunden werden, relevante Szenarien für die Generierung von Testfällen zu identifizieren, um die Sicherung von hochautomatisierten \ac{FAS} mit ökonomischen Aufwand garantieren zu können.

Genau hier soll diese Arbeit einen Beitrag leisten. Das Ziel, wie bereits in Abschnitt \ref{einleitung_zielsetzung} erläutert, ist die Identifikation von bisher unbekannten Fahrszenarien. Die Grundidee ist es einen Klassifikator mit einem großen Anteil synthetischer Daten und einem kleinen Anteil realer Daten von bisher bekannten Szenarien zu trainieren. Dieser Klassifikator kann dann bekannte Szenarien erkennen, liefert aber keine eindeutigen Ergebnisse bei bisher unbekannten Szenarien. Mit dieser Methodik soll es möglich sein bisher unbekannte Fahrszenarien zu identifizieren, um auf der Basis neue Testfälle für die Sicherung hochautomatisierter Fahrfunktionen zu generieren.

\begin{figure}[h]
\centering
\includegraphics[scale=0.5]{teilmenge_fahrszenarien.pdf}
\caption{Beziehung zwischen bekannten und unbekannten Fahrszenarien}
\label{fig_teilmenge_fahrszenarien}
\end{figure}

In dieser Arbeit soll ein Proof-of-Concept für diese Methodik entwickelt werden. Dafür wird im folgenden Abschnitt \ref{konzept_struktur} das Konzept im Detail und die Vorgehensweise vorgestellt. Anschließend wird in Abschnitt \ref{konzept_methodik} die Methodik erklärt mit welcher dieses Konzept umgesetzt werden soll.

- funktionale Dekomposition

% ===========================
\section{Struktur}
\label{konzept_struktur}
% ===========================

- Definition von Fahrszenarien
- Simulation von Fahrten mit diesen Szenarien
- Extraktion und Labeling der Fahrszenarien
- Extraktion von Reale Fahrszenarien
- Training NN mit synthetischen und realen
- Evaluation mit realen Fahrszenarien

Im ersten Schritt der Umsetzung werden bestimmte Fahrszenarien ausgewählt und, wie in Abschnitt \ref{grundlagen_fahren_szenarien} erläutert, definiert. In dieser Arbeit werden Szenarien auf der Ebene der \textit{logischen Szenarien} definiert. Für das Training eines Klassifikators werden im nächsten Schritt synthetische und reale Daten benötigt.

Für die Generierung von synthetischen Daten 



Gleichzeitig bleibt der Aufwand für das Labeling der Daten überschaubar, weil die synthetischen Daten automatisiert gelabelt werden können und die realen Daten nur einen geringen Anteil ausmachen sollen.

% ===========================
\section{Methodik}
\label{konzept_methodik}
% ===========================

Lorem ipsum dolor sit amet, consetetur sadipscing elitr, sed diam nonumy eirmod tempor invidunt ut labore et dolore magna aliquyam erat, sed diam voluptua. At vero eos et accusam et justo duo dolores et ea rebum. Stet clita kasd gubergren, no sea takimata sanctus est Lorem ipsum dolor sit amet.  







 
\input{chapters/datengenerierung}

\input{chapters/training}


% ===========================
\chapter{Evaluation}
\label{evaluation}
% ===========================

Nam liber tempor cum soluta nobis eleifend option congue nihil imperdiet doming id quod mazim placerat facer possim assum. Lorem ipsum dolor sit amet, consectetuer adipiscing elit, sed diam nonummy nibh euismod tincidunt ut laoreet dolore magna aliquam erat volutpat. Ut wisi enim ad minim veniam, quis nostrud exerci tation ullamcorper suscipit lobortis nisl ut aliquip ex ea commodo consequat.


% ===========================
\section{Simulierte Daten}
\label{evaluation_simuliert}
% ===========================

Nam liber tempor cum soluta nobis eleifend option congue nihil imperdiet doming id quod mazim placerat facer possim assum. Lorem ipsum dolor sit amet, consectetuer adipiscing elit, sed diam nonummy nibh euismod tincidunt ut laoreet dolore magna aliquam erat volutpat. Ut wisi enim ad minim veniam, quis nostrud exerci tation ullamcorper suscipit lobortis nisl ut aliquip ex ea commodo consequat.


% ===========================
\section{Reale Daten}
\label{evaluation_real}
% ===========================

Nam liber tempor cum soluta nobis eleifend option congue nihil imperdiet doming id quod mazim placerat facer possim assum. Lorem ipsum dolor sit amet, consectetuer adipiscing elit, sed diam nonummy nibh euismod tincidunt ut laoreet dolore magna aliquam erat volutpat. Ut wisi enim ad minim veniam, quis nostrud exerci tation ullamcorper suscipit lobortis nisl ut aliquip ex ea commodo consequat.

\input{chapters/conclusion}

\cleardoublepage
\pagenumbering{Roman}
\setcounter{page}{11}

\renewcommand{\bibname}{Literaturverzeichnis}
\printbibliography
\addcontentsline{toc}{chapter}{Literaturverzeichnis}

\printglossary[type=\acronymtype, title={Abkürzungsverzeichnis}]
\addcontentsline{toc}{chapter}{Abkürzungsverzeichnis}

\listoftables
\addcontentsline{toc}{chapter}{Tabellenverzeichnis}

\listoffigures
\addcontentsline{toc}{chapter}{Abbildungsverzeichnis}

\appendix


% ===========================
\chapter*{Appendix}
\label{appendix}
% ===========================

\begin{flushleft}

% ===========================
\section*{Generierte Daten dieser Arbeit}
\label{appendix_data}
% ===========================

Unter dem nachfolgenden Link sind alle generierten Daten dieser Arbeit verfügbar:

\vspace{0,5cm}
\url{https://bit.ly/2A6hMKH}
\vspace{0,5cm}

Im Ordner \textit{Raw Data CarMaker} sind die generierten Signal- und Bilddaten aus der Simulationssoftware CarMaker verfügbar. Im Ordner \textit{Processed Data} sind alle synthetischen und realen annotierten Szenarien als Numpy-Arrays gespeichert. Die trainierten \acp{KNN}-Modelle und alle dazugehörigen Ergebnisdaten sind im Ordner \textit{Trained Neural Networks} zu finden.


% ===========================
\section*{Python-Code dieser Arbeit}
\label{appendix_code}
% ===========================

Unter dem nachfolgenden Link ist der gesamte Python-Code, der in dieser Arbeit geschrieben wurde, zu finden:

\vspace{0,5cm}
\url{https://github.com/manuelhk/master-thesis-python}
\vspace{0,5cm}

Der gesamte Code ist in der Datei \textit{README.md} beschrieben.

\end{flushleft}


\end{document}
