\documentclass[12pt, twoside, ngerman, openright]{report}

\usepackage[utf8]{inputenc}
\usepackage{graphicx}
\usepackage{subfig}
\usepackage[top=25mm, bottom=25mm, bindingoffset=6mm, includeheadfoot]{geometry}
\usepackage{fancyhdr}
\usepackage{emptypage}
\usepackage[style=alphabetic]{biblatex}
\usepackage[ngerman]{babel}
\usepackage[printonlyused]{acronym}
\usepackage[onehalfspacing]{setspace}
\usepackage{titlesec}
\usepackage{todonotes}
\usepackage{longtable}
\usepackage{listings}
\usepackage{color}
\usepackage{amsmath}
\usepackage{pdfpages}
\usepackage{hyperref}


% path to bibliography
\addbibresource{chapters/my_bib.bib}

% path to images
%\graphicspath{{images/}}


% design header and footer of all pages
\pagestyle{fancy}
\renewcommand{\chaptermark}[1]{\markboth{\thechapter.\ #1}{}}
\renewcommand{\sectionmark}[1]{\markright{\thesection.\ #1}{}}
\fancyhead{}
\fancyhead[RE]{\leftmark}
\fancyhead[LO]{\rightmark}
\fancyfoot{}
\fancyfoot[RE]{\footnotesize{Institut für Technik der Informationsverarbeitung\\
			   Karlsruher Institut für Technologie}}
\fancyfoot[LO]{\footnotesize{\mytitle}}
\fancyfoot[LE,RO]{\thepage}
\renewcommand{\headrulewidth}{0.4pt}
\renewcommand{\footrulewidth}{0.4pt}
\fancypagestyle{plain}{
	\fancyhf{} % clear all header and footer fields
	\fancyfoot[LE,RO]{\thepage}
	\fancyfoot[LO]{\footnotesize{\mytitle}}
	\renewcommand{\headrulewidth}{0pt}
	\renewcommand{\footrulewidth}{0.4pt}}


% remove "Kapitel X" from every chapter headline
%\titleformat{\chapter}[display]
%  {\Huge\bfseries}
%  {}
%  {0pt}
%  {\thechapter.\ }

%\titleformat{name=\chapter,numberless}[display]
%  {\Huge\bfseries}
%  {}
%  {0pt}
%  {}


% remove "+" from citations
\renewcommand*{\labelalphaothers}{}


% definition of own variables
\newcommand{\submissiondate}{21. Dezember 2018}
\newcommand{\mytitle}{Künstliches Lernen: Trainingsdaten aus dem virtuellen Fahrversuch \\ für Szenarienklassifizierung mit Deep Learning Algorithmen}

% define style/settings of code
\definecolor{codegreen}{rgb}{0,0.6,0}
\definecolor{codegray}{rgb}{0.5,0.5,0.5}
\definecolor{codeorange}{rgb}{0.99,0.4,0}
\definecolor{backcolour}{rgb}{0.95,0.95,0.95} 
\lstdefinestyle{mystyle}{
    backgroundcolor=\color{backcolour},   
    commentstyle=\color{codegreen},
    keywordstyle=\color{codeorange},
    numberstyle=\tiny\color{codegray},
    stringstyle=\color{codegray},
    basicstyle=\footnotesize,
    breakatwhitespace=false,         
    breaklines=true,                 
    captionpos=b,                    
    keepspaces=true,                 
    numbers=left,                    
    numbersep=5pt,                  
    showspaces=false,                
    showstringspaces=false,
    showtabs=false,                  
    tabsize=2
}
\lstset{style=mystyle}



\begin{document}

\pagestyle{fancy}
\pagenumbering{roman}

%
\begin{titlepage}
    \begin{center}
        \vspace*{1cm}
 
        \Huge
        \textbf{\mytitle}
 
        \vspace{0.5cm}
        \LARGE
        Thesis Subtitle
 
        \vspace{1.5cm}
 
        \textbf{Author Name}
 
        \vfill
 
        A thesis presented for the degree of\\
        Master of Science
 
        \vspace{0.8cm}
 
        \includegraphics[width=0.4\textwidth]{KITLogo}
 
        \Large
        Department Name\\
        University Name\\
        Country\\
        Date
 
    \end{center}
\end{titlepage}
\includepdf[pages=1]{chapters/titelseite_itiv.pdf}

%\cleardoublepage


\chapter*{Abstract}

Abstract \gls{itiv} goes \gls{t} here \cite{einstein}. test mit Ö, ä und ß ...
%\addcontentsline{toc}{chapter}{Zusammenfassung}

%\cleardoublepage


\chapter*{Eidesstattliche Erklärung}

Ich versichere wahrheitsgemäß, die Arbeit selbstständig verfasst, alle benutzten Hilfsmittel vollständig und genau angegeben und alles kenntlich gemacht zu haben, was aus Arbeiten anderer unverändert oder mit Abänderungen entnommen wurde sowie die Satzung des KIT zur Sicherung guter wissenschaftlicher Praxis in der jeweils gültigen Fassung beachtet zu haben.\\

\vspace{1cm}

\noindent Karlsruhe, \submissiondate

\vspace{2.5cm}

\noindent\rule{5cm}{0.4pt} \\
Manuel Kaiser \\
%\addcontentsline{toc}{chapter}{Eidesstattliche Erklärung}

\tableofcontents

%\cleardoublepage


\chapter*{Abkürzungsverzeichnis}

\todo{Items einrücken wie andere Verzeichnisse}

\begin{acronym}
\acro{MiL}{Model-in-the-Loop}
\acro{SiL}{Software-in-the-Loop}
\acro{HiL}{Hardware-in-the-Loop}
\acro{ViL}{Vehicle-in-the-Loop}
\acro{XiL}{X-in-the-Loop}
\acro{FAS}{Fahrerassistenzsysteme}
\acro{PCA}{Principal Component Analysis}
\acro{RF}{Random Forest}
\acro{SVM}{Support Vector Machine}
\acro{FRC}{Fuzzy Rule-Based Classifier}
\acro{kNN}{k-Nearest-Neighbor}
\acro{HMM}{Hidden Markov Model}
\acro{KNN}{künstliches neuronales Netz}
\acroplural{KNN}[KNNs]{künstliche neuronale Netze}
\acro{DNN}{Deep Neural Network}
\acro{RNN}{Recurrent Neural Network}
\acro{CNN}{Convolutional Neural Network}
\acro{LSTM}{Long Short-Term Memory}
\acro{tanh}{Tangens Hyperbolicus}
\acro{ReLU}{Rectified Linear Unit}
\end{acronym}

\addcontentsline{toc}{chapter}{Abkürzungsverzeichnis}

%\cleardoublepage
\listoffigures
\addcontentsline{toc}{chapter}{Abbildungsverzeichnis}

%\cleardoublepage
\listoftables
\addcontentsline{toc}{chapter}{Tabellenverzeichnis}

\cleardoublepage
\pagenumbering{arabic}


% ===========================
\chapter{Einleitung}
\label{einleitung}
% ===========================


% ===========================
\section{Problemstellung und Motivation}
\label{einleitung_problemstellung}
% ===========================

Hochautomatisierte Fahrerassistenzsysteme werden zunehmend komplexer. Herkömmliche Testmethoden sind durch die Vielzahl an möglichen Szenarien nicht mehr praktisch testbar. Heutzutage wird schon vieles in Simulation getestet. Dabei git es aktuell noch Probleme..

Autonomes Fahren - kaum ein Trend ist aktuell ein stärkerer Treiber in der Automobilindustrie. Dabei spielt der Einsatz von Verfahren des maschinellen Lernens ein bedeutende Rolle. Eine große Herausforderung für diese Algorithmen ist, dass Trainingsdaten, sofern sie auf realen, aufgezeichneten Daten beruhen, manuell annotiert werden müssen, was diesen Prozess sehr aufwändig macht. Ein weiteres Problem von realen Daten ist Ihre geringe Varianz. Während Standardsituationen sehr häufig vorkommen und damit auch mit einem neuronalen Netz erlernt werden können, gibt es einige Situation die selten auftreten, allerdings sehr kritisch sind. Es ist daher schwieriger ein neuronales Netz für dieses Situationen, wie z.B. das „schneiden“ eines anderen Fahrzeugs beim Spurwechsel, zu trainieren.

Genau hier soll diese Arbeit ansetzen. Es soll ein Konzept entwickelt und umgesetzt werden, wie eine bereits existierende Simulationsumgebung eingesetzt werden kann, um neuronale Netze zu trainieren und zu testen.


% ===========================
\section{Zielsetzung}
\label{einleitung_zielsetzung}
% ===========================

Das Ergebnis der Arbeit soll eine Methodik sein bisher unbekannte Testfälle zu finden. Dabei sollen Videodaten mit CarMaker erzeugt und mit diesen Daten ein neuronales Netz trainiert werden um Fahrszenarien zu klassifizieren. Mit einem trainierten neuronalen Netz sollen auch Fahrszenarien mit realen Daten erkannt und klassifiziert werden.

Die oben genannten Probleme sollen mit der Verwendung von simulierten Trainingsdaten adressiert und weiter untersucht werden:

\begin{itemize}
\item Trainingsdaten müssen nicht mehr aufwendig manuell annotiert werden.
\item Die Umgebung ist bei der Simulation der Daten vollständig kontrollierbar und die Datenerfassung wird effizienter.
\item Bisher unbekannte Testfälle können gefunden werden.
\end{itemize}

Diese Arbeit soll einen theoretischen und praktischen Beitrag zum automatisierten Training von neuronalen Netzen im Bereich automatisiertem Fahren liefern. Der Fokus liegt dabei auf den Möglichkeiten und Herausforderungen, die sich durch die Verwendung von simulierten Trainingsdaten ergeben.




% ===========================
\chapter{Grundlagen}
\label{grundlagen}
% ===========================

Das Ziel dieses Kapitels ist es ... in italic


% ===========================
\section{Hochautomatisierte Fahren}
\label{grundlagen_fahren}
% ===========================


% ===========================
\subsection{Entwicklung hochautomatisierter Fahrfunktionen}
\label{grundlagen_fahren_entwicklung}
% ===========================

Lorem ipsum dolor sit amet, consetetur sadipscing elitr, sed diam nonumy eirmod tempor invidunt ut labore et dolore magna aliquyam erat, sed diam voluptua. At vero eos et accusam et justo duo dolores et ea rebum. Stet clita kasd gubergren, no sea takimata sanctus est Lorem ipsum dolor sit amet. Lorem ipsum dolor sit amet, consetetur sadipscing elitr, sed diam nonumy eirmod tempor invidunt ut labore et dolore magna aliquyam erat, sed diam voluptua. At vero eos et accusam et justo duo dolores et ea rebum. Stet clita kasd gubergren, no sea takimata sanctus est Lorem ipsum dolor sit amet. Lorem ipsum dolor sit amet, consetetur sadipscing elitr, sed diam nonumy eirmod tempor invidunt ut labore et dolore magna aliquyam erat, sed diam voluptua. At vero eos et accusam et justo duo dolores et ea rebum. Stet clita kasd gubergren, no sea takimata sanctus est Lorem ipsum dolor sit amet. 

Lorem ipsum dolor sit amet, consetetur sadipscing elitr, sed diam nonumy eirmod tempor invidunt ut labore et dolore magna aliquyam erat, sed diam voluptua. At vero eos et accusam et justo duo dolores et ea rebum. Stet clita kasd gubergren, no sea takimata sanctus est Lorem ipsum dolor sit amet. 


% ===========================
\subsection{Simulation in der Entwicklung}
\label{grundlagen_fahren_simulation}
% ===========================

Lorem ipsum dolor sit amet, consetetur sadipscing elitr, sed diam nonumy eirmod tempor invidunt ut labore et dolore magna aliquyam erat, sed diam voluptua. At vero eos et accusam et justo duo dolores et ea rebum. Stet clita kasd gubergren, no sea takimata sanctus est Lorem ipsum dolor sit wie in Graph \ref{fig:x cubed graph} beschrieben.Lorem ipsum dolor sit amet, consetetur sadipscing elitr, sed diam nonumy eirmod tempor invidunt ut labore et dolore magna aliquyam erat, sed diam voluptua. At vero eos et accusam et justo duo dolores et ea rebum. Stet clita kasd gubergren, no sea takimata sanctus est Lorem ipsum dolor sit wie in Graph \ref{fig:x cubed graph} beschrieben.

Lorem ipsum dolor sit amet, consetetur sadipscing elitr, sed diam nonumy eirmod tempor invidunt ut labore et dolore magna aliquyam erat, sed diam voluptua. At vero eos et accusam et justo duo dolores et ea rebum. Stet clita kasd gubergren, no sea takimata sanctus est Lorem ipsum dolor sit wie in Graph \ref{fig:x cubed graph} beschrieben.


\begin{figure}[h]
\centering
\includegraphics[scale=0.5]{graph_a}
\caption{An example graph}
\label{fig:x cubed graph}
\end{figure}

Duis autem vel eum iriure dolor in hendrerit in vulputate velit esse molestie consequat, vel illum dolore eu feugiat nulla facilisis at vero eros et accumsan et iusto odio dignissim qui blandit praesent luptatum zzril delenit augue duis dolore te feugait nulla facilisi.   


% ===========================
\section{Neuronale Netze}
\label{grundlagen_nn}
% ===========================

Duis autem vel eum iriure dolor in hendrerit in vulputate velit esse molestie consequat, vel illum dolore eu feugiat nulla facilisis.


% ===========================
\subsection{Maschinelles Lernen}
\label{grundlagen_nn_ml}
% ===========================

Duis autem vel eum iriure dolor in hendrerit in vulputate velit esse molestie consequat, vel illum dolore eu feugiat nulla facilisis at vero eros et accumsan et iusto odio dignissim qui blandit praesent luptatum zzril delenit augue duis dolore te feugait nulla facilisi. 


% ===========================
\subsection{Convolutional Neural Network}
\label{grundlagen_nn_cnn}
% ===========================


\begin{table}[h]
\centering
\begin{tabular}{l | l | l}
A & B & C \\
\hline
1 & 2 & 3 \\
4 & 5 & 6
\end{tabular}
\caption{very basic table}
\label{tab:abc}
\end{table}


% ===========================
\subsection{Recurrent Neural Network}
\label{grundlagen_nn_rnn}
% ===========================

Lorem ipsum dolor sit amet, consectetuer adipiscing elit, sed diam nonummy nibh euismod tincidunt ut laoreet dolore magna aliquam erat volutpat \cite{latexcompanion}. 




% ===========================
\chapter{Konzept}
\label{konzept}
% ===========================

Wie in Abschnitt \ref{grundlagen_fahren} beschrieben, stellt Sicherung von hochautomatisierten \ac{FAS} die Automobilindustrie vor große Herausforderungen. Die Menge der bekannten Fahrszenarien ist nur eine Teilmenge aller Szenarien, die zukünftige \ac{FAS} abdecken müssen. Diese Beziehung ist schematisch in Abbildung \ref{fig_teilmenge_fahrszenarien} dargestellt. Die Folge ist eine steigende Anzahl benötigter Testkilometer, die in Zukunft mit ökonomischem Aufwand nicht mehr umsetzbar sein wird. Es müssen neue Methoden gefunden werden, relevante Szenarien für die Generierung von Testfällen zu identifizieren, um die Sicherung von hochautomatisierten \ac{FAS} mit ökonomischen Aufwand garantieren zu können.

Genau hier soll diese Arbeit einen Beitrag leisten. Das Ziel, wie bereits in Abschnitt \ref{einleitung_zielsetzung} erläutert, ist die Identifikation von bisher unbekannten Fahrszenarien. Die Grundidee ist es einen Klassifikator mit einem großen Anteil synthetischer Daten und einem kleinen Anteil realer Daten von bisher bekannten Szenarien zu trainieren. Dieser Klassifikator kann dann bekannte Szenarien erkennen, liefert aber keine eindeutigen Ergebnisse bei bisher unbekannten Szenarien. Mit dieser Methodik soll es möglich sein bisher unbekannte Fahrszenarien zu identifizieren, um auf der Basis neue Testfälle für die Sicherung hochautomatisierter Fahrfunktionen zu generieren.

\begin{figure}[h]
\centering
\includegraphics[scale=0.5]{teilmenge_fahrszenarien.pdf}
\caption{Beziehung zwischen bekannten und unbekannten Fahrszenarien}
\label{fig_teilmenge_fahrszenarien}
\end{figure}

In dieser Arbeit soll ein Proof-of-Concept für diese Methodik entwickelt werden. Dafür wird im folgenden Abschnitt \ref{konzept_struktur} das Konzept im Detail und die Vorgehensweise vorgestellt. Anschließend wird in Abschnitt \ref{konzept_methodik} die Methodik erklärt mit welcher dieses Konzept umgesetzt werden soll.

- funktionale Dekomposition

% ===========================
\section{Struktur}
\label{konzept_struktur}
% ===========================

- Definition von Fahrszenarien
- Simulation von Fahrten mit diesen Szenarien
- Extraktion und Labeling der Fahrszenarien
- Extraktion von Reale Fahrszenarien
- Training NN mit synthetischen und realen
- Evaluation mit realen Fahrszenarien

Im ersten Schritt der Umsetzung werden bestimmte Fahrszenarien ausgewählt und, wie in Abschnitt \ref{grundlagen_fahren_szenarien} erläutert, definiert. In dieser Arbeit werden Szenarien auf der Ebene der \textit{logischen Szenarien} definiert. Für das Training eines Klassifikators werden im nächsten Schritt synthetische und reale Daten benötigt.

Für die Generierung von synthetischen Daten 



Gleichzeitig bleibt der Aufwand für das Labeling der Daten überschaubar, weil die synthetischen Daten automatisiert gelabelt werden können und die realen Daten nur einen geringen Anteil ausmachen sollen.

% ===========================
\section{Methodik}
\label{konzept_methodik}
% ===========================

Lorem ipsum dolor sit amet, consetetur sadipscing elitr, sed diam nonumy eirmod tempor invidunt ut labore et dolore magna aliquyam erat, sed diam voluptua. At vero eos et accusam et justo duo dolores et ea rebum. Stet clita kasd gubergren, no sea takimata sanctus est Lorem ipsum dolor sit amet.  







 

% ===========================
\chapter{Umsetzung}
\label{umsetzung}
% ===========================

Lorem ipsum dolor sit amet, consetetur sadipscing elitr, sed diam nonumy eirmod tempor invidunt ut labore et dolore magna aliquyam erat, sed diam voluptua.


\begin{lstlisting}[language=Python]
import numpy as np
# This is a comment
a = 5
for x in range(a):
   print("Hello")
\end{lstlisting}


Dropout - \cite{hinton2012improving}


% ===========================
\section{Definition der Fahrszenarien}
\label{umsetzung_definition}
% ===========================

In diesem Abschnitt werden Szenarien, wie in Abschnitt \ref{grundlagen_fahren_szenarien} beschrieben, als \textit{logische Szenarien} für das weitere Vorgehen in dieser Arbeit definiert. In Anlehnung an bestehende Arbeiten zu Erkennung von Fahrszenarien und auf Basis von Machbarkeitsabschätzungen für die Umsetzung werden in dieser Arbeit die Szenarien \textit{free cruising}, \textit{following}, \textit{catching up}, \textit{lane change left} und \textit{lane change right} auf der Autobahn betrachtet. Die Autobahn wurde ausgewählt, weil es weniger Parameter zu betrachten gibt als auf anderen Straßen wie beispielsweise in der Stadt. In der folgenden Tabelle \ref{tab_definition_szenarios} werden diese Szenarien auf \textit{funktionaler} und \textit{logischer Ebene} definiert.

Um die Darstellung in der Tabelle zu erleichtern werden folgende Abstände zwischen Ego-Fahrzeug und Fahrzeug 2 definiert. Dabei beschreibt $ego_v$ die Geschwindigkeit des Ego-Fahrzeugs in [km/h].

\begin{equation}
s_0 = ego_v * h \qquad [m]
\end{equation}

\begin{equation}
s_1 = ego_v * h * \frac{2}{3} \qquad [m]
\end{equation}

\begin{equation}
s_2 = ego_v * h * \frac{1}{3} \qquad [m]
\end{equation}

\small
\begin{longtable}[c]{p{1.5cm} p{6cm} p{6cm}}
\textbf{Szenario} & \textbf{Funktionale Definition} & \textbf{Logische Definition} \\
\hline
\endhead

\textbf{Alle} & 2-spurige Autobahn geradeaus oder in einer Kurve, Geschwindigkeitsbegrenzung ist größer als 80 km/h & Breite Fahrstreifen [2,3..3,5] m \newline Geschwindigkeitsbegrenzung [80..keine] km/h \\
\hline

\textbf{Alle} & Tageslicht, keine Wolken bis leicht bewölkt, kein Niederschlag, gute Sichtbedingungen & Tageszeit [Sonnenaufgang..Sonnenuntergang] \newline Bewölkung [leicht bewölkt..wolkenlos]\\
\hline \hline

\textbf{Free Cruising} & Ego, andere Verkehrsteilnehmer \newline \underline{Interaktion:} Ego fährt frei auf linker oder rechter Fahrspur, andere Fahrzeuge sind weit entfernt und haben keinen Einfluss auf die Manöver des Ego & Geschwindigkeit Ego [60..200] km/h \newline Abstand zu anderen Verkehrsteilnehmern [$>s_0$] m \\
\hline

\textbf{Following} & Ego, andere Verkehrsteilnehmer \newline \underline{Interaktion:} Ego fährt auf linker oder rechter Fahrspur in sicherem Abstand hinter Fahrzeug 2 & Geschwindigkeit Ego [60..200] km/h \newline Geschwindigkeitsdifferenz zwischen Ego und Fahrzeug 2  [$<ego_v*0,05$] km/h \newline Abstand Ego zu Fahrzeug 2 [$s_2$..$s_1$] m \newline Ego befindet sich auf gleicher Fahrspur hinter Fahrzeug 2 \\
\hline

\textbf{Catching up} & Ego, andere Verkehrsteilnehmer \newline \underline{Interaktion:} Ego fährt auf linker Fahrspur und verringert den vertikalen Abstand zu Fahrzeug 2 auf der rechten Fahrspur (auf-/überholen) & Geschwindigkeit Ego [60..200] km/h \newline Geschwindigkeit Fahrzeug 2 [$<ego_v$] \newline Vertikaler Abstand Ego zu Fahrzeug 2 [0..$s_0$] m \newline Ego fährt auf linker Fahrspur hinter Fahrzeug 2 das auf rechter Fahrspur fährt \\
\hline

\textbf{Lane change left} & Ego, andere Verkehrsteilnehmer sind optional \newline \underline{Interaktion:} Ego fährt auf rechter Fahrspur und wechselt auf linke Fahrspur & Geschwindigkeit Ego [60..200] km/h \newline Ego befindet sich auf rechter Fahrspur und wechselt auf linke Fahrspur \\
\hline

\textbf{Lane change right} & Ego, andere Verkehrsteilnehmer sind optional \newline \underline{Interaktion:} Ego fährt auf linker Fahrspur und wechselt auf rechte Fahrspur & Geschwindigkeit Ego [60..200] km/h \newline Ego befindet sich auf linker Fahrspur und wechselt auf rechte Fahrspur \\
\hline

\caption{Definition der Szenarien \textit{free cruising}, \textit{following}, \textit{catching up}, \textit{lane change left} und \textit{lane change right}}
\label{tab_definition_szenarios}
\end{longtable}
\normalsize

% ===========================
\section{Generierung synthetischer Daten}
\label{umsetzung_daten_synth}
% ===========================

Auf Basis der Definitionen aus dem vorherigen Abschnitt \ref{umsetzung_definition} werden in diesem Abschnitt die benötigten Signal- und Bilddaten simuliert und entsprechend gelabelt. Dafür werden in Abschnitt \ref{umsetzung_daten_synth_simulation} die Signaldaten, die für die eindeutige Klassifizierung der Szenarien benötigt werden, simuliert. In Abschnitt \ref{umsetzung_daten_synth_labeling} werden diese Signaldaten verwendet um die parallel simulierten Bilddaten entsprechend zu labeln.
 
% ===========================
\subsection{Simulation mit CarMaker}
\label{umsetzung_daten_synth_simulation}
% ===========================

Für die Simulation der Signal- und Bilddaten wird die kommerzielle Software CarMaker von IPG Automotive \cite{ipg2018carmaker} verwendet. Diese Simulationssoftware wird für den virtuellen Fahrversuch und \ac{HiL}-Tests eingesetzt um Komponenten in unterschiedlichen Szenarien zu testen. In dieser Arbeit wird CarMaker verwendet um die Szenarien \textit{free cruising}, \textit{following}, \textit{catching up}, \textit{lane change left} und \textit{lane change right} zu simulieren. 

Für die Aufnahme der benötigten Bilddaten wird im simulierten Fahrzeug ein entsprechender Kamerasensor konfiguriert. Die Konfiguration des Sensors orientiert sich an der Konfiguration von realen Frontview-Kameras im Fahrzeug nach Punke \cite{punke2015kamera}. So ist der Kamerasensor an der Stelle des Rückfahrspiegels platziert und hat eine Auflösung von 640x480 Pixeln, ein vertikales Blickfeld von XX° und ein horizentales Blickfeld von XX° \todo{Konfiguration einfügen}. Die Konfiguration und Position des Sensors im Modul \textit{CarMaker - Vehicle Data Set} ist in Abbildung \ref{fig_car_camera_sensor} zu sehen.

\begin{figure}[h]
\centering
\includegraphics[scale=0.4]{car_camera_sensor.png}
\caption{Konfiguration des Kamerasensors im Modul \textit{CarMaker - Vehicle Data Set} \cite{ipg2018carmaker}}
\label{fig_car_camera_sensor}
\end{figure}

Die benötigten Signaldaten für das Labeln werden von den Definitionen aus Abschnitt \ref{umsetzung_definition} abgeleitet. Für die eindeutige Identifikation der logischen Szenarien werden die folgenden Werte benötigt: Geschwindigkeit des Ego-Fahrzeugs, Abstand und Geschwindigkeitsdifferenz des Ego-Fahrzeugs zu allen anderen Fahrzeugen, aktuelle Fahrspur des Ego-Fahrzeugs und allen anderen Fahrzeugen und die relative Position des Ego-Fahrzeugs, i.e. ob sich das Ego-Fahrzeug vor oder hinter einem anderen Fahrzeug befindet. Um den Abstand und die Geschwindigkeitsdifferenz des Ego-Fahrzeugs zu allen anderen Fahrzeugen aufzuzeichnen, wird ein Objektsensor im Ego-Fahrzeug konfiguriert. Mit diesem Sensor können im konfigurierten Radius alle Fahrzeuge und ihr Abstand und ihre relative Geschwindigkeit zum Ego-Fahrzeug erfasst und über die \textit{OutputQuantities} in CarMaker aufgezeichnet werden. Die Geschwindigkeit des Ego-Fahrzeugs und die Fahrspur-ID und Position aller Fahrzeuge können direkt, ohne zusätzlichen Sensor, über die \textit{OutputQuantities} aufgezeichnet werden. Die Werte und ihre jeweilige Variable in CarMaker sind in der Tabelle \ref{tab_output_quantities} zusammengefasst.

\begin{table}[h]
\small
\centering
\def\arraystretch{1.4}
\begin{tabular}{p{6.2cm} p{7.5cm}}
\textbf{Variable in CarMaker} & \textbf{Beschreibung} \\
\hline

Car.v & Geschwindigkeit des Ego-Fahrzeugs in [m/s] \\
Car.Road.sRoad & Position des Ego-Fahrzeugs auf der Strecke in [m] \\
Car.Road.Lane.Act.LaneId & Fahrspur-ID des Ego-Fahrzeugs \\
\hline
Sensor.Object.OB01.TX.NearPnt.dv\_p & Geschwindigkeitsdifferenz zwischen Fahrzeug TX und dem Ego-Fahrzeug in [m/s] \\
Sensor.Object.OB01.TX.NearPnt.ds\_p & Abstand zwischen Fahrzeug TX und dem Ego-Fahrzeug in [m] \\
\hline
Traffic.TX.sRoad & Position des Fahrzeugs TX auf der Strecke in [m] \\
Traffic.TX.Lane.Act.LaneId & Fahrspur-ID des Fahrzeugs TX \\
\hline

\end{tabular}
\caption{Aufgezeichnete Signaldaten in CarMaker}
\label{tab_output_quantities}
\end{table}

Für die Simulation werden zwei Strecken der Länge 6.000 m und 10.000 m mit dem \textit{CarMaker - Scenario Editor} erstellt. Bei beiden Strecken handelt es sich um eine 4-spurige Autobahn mit zwei Fahrspuren in jede Richtung. Die Fahrtrichtungen sind in der Mitte von einer Leitplanke getrennt und am Rand der Fahrbahn sind jeweils Standstreifen vorhanden. Abschnittsweise stehen neben der Fahrbahn auch einige Bäume, was in Abbildung \ref{fig_cm_road_strecke} mit grünen Streifen gekennzeichnet ist. Abbildung \ref{fig_cm_road_bild} zeigt die Konfiguration der simulierten Straße \todo{Bild einfügen}.

\begin{figure}[h]
\centering
\includegraphics[scale=0.4]{cm_road_bild.jpg}
\caption{Konfiguration der simulierten Straße \cite{ipg2018carmaker}}
\label{fig_cm_road_bild}
\end{figure}

\begin{figure}[h]
\centering
\begin{tabular}{c}
\subfloat[Strecke 1]{\includegraphics[scale=0.8]{cm_road_1.png}} \\
\subfloat[Strecke 2]{\includegraphics[scale=0.8]{cm_road_2.png}}
\end{tabular}
\caption{Schema der simulierten Strecken 1 und 2 \cite{ipg2018carmaker}}
\label{fig_cm_road_strecke}
\end{figure}

Die Simulation und Generierung von Bild- und Signaldaten wird mit dem \textit{CarMaker - Test Manager} durchgeführt. Mit diesem Modul lassen sich Fahrten mit unterschiedlichen Konfigurationen simulieren. In dieser Arbeit werden die Variablen \textit{Geschwindigkeit}, \textit{Mindestabstand zu vorausfahrendem Fahrzeug}, \textit{Aggressivität beim Überholen} und \textit{Minimale Geschwindigkeitsdifferenz beim Überholen} auf beiden oben beschriebenen Strecken variiert. Die Werte der Variablen, die simuliert werden, sind in Tabelle \ref{tab_tm_variablen} aufgelistet und sind aus die Sicht des Ego-Fahrzeugs.

\begin{table}[h]
\small
\centering
\def\arraystretch{1.4}
\begin{tabular}{p{8cm} p{0.7cm} p{0.7cm} p{0.7cm} p{0.7cm} p{0.7cm}}
%\multicolumn{2}{|c|}{\textbf{TITLE}}
\textbf{Variable} & \textbf{Werte} & & & & \\
\hline
Geschwindigkeit in [km/h] & 100 & 120 & 140 & 160 & 180 \\
Mindestabstand zu vorausfahrendem Fahrzeug in [s] & 1,0 & 1,5 & 2,0 & & \\
Aggressivität beim Überholen & 0,2 & 0,6 & 1,0 & & \\
Minimale Geschwindigkeitsdifferenz beim Überholen in [km/h] & 5 & 15 & 25 & & \\
\hline
\end{tabular}
\caption{Variablen und Werte die in der Simulation verwendet werden}
\label{tab_tm_variablen}
\end{table}





% ===========================
\subsection{Daten Labeling}
\label{umsetzung_daten_synth_labeling}
% ===========================

Duis autem vel eum iriure dolor in hendrerit in vulputate velit esse molestie consequat, vel illum dolore eu feugiat nulla facilisis at vero eros et accumsan et iusto odio dignissim qui blandit praesent luptatum zzril delenit augue duis dolore te feugait nulla facilisi. Lorem ipsum dolor sit amet, consectetuer adipiscing elit, sed diam nonummy nibh euismod tincidunt ut laoreet dolore magna aliquam erat volutpat.


% ===========================
\section{Generierung realer Daten}
\label{umsetzung_daten_real}
% ===========================

Ut wisi enim ad minim veniam, quis nostrud exerci tation ullamcorper suscipit lobortis nisl ut aliquip ex ea commodo consequat.


% ===========================
\section{Training}
\label{umsetzung_training}
% ===========================

Nam liber tempor cum soluta nobis eleifend option congue nihil imperdiet doming id quod mazim placerat facer possim assum. Lorem ipsum dolor sit amet, consectetuer adipiscing elit, sed diam nonummy nibh euismod tincidunt ut laoreet dolore magna aliquam erat volutpat.


% ===========================
\subsection{Inputdaten}
\label{umsetzung_training_input}
% ===========================

Ut wisi enim ad minim veniam, quis nostrud exerci tation ullamcorper suscipit lobortis nisl ut aliquip ex ea commodo consequat. Duis autem vel eum iriure dolor in hendrerit in vulputate velit esse molestie consequat, vel illum dolore eu feugiat nulla facilisis at vero eros et accumsan et iusto odio dignissim qui blandit praesent luptatum zzril delenit augue duis dolore te feugait nulla facilisi.


% ===========================
\subsection{Architektur des neuronalen Netzes}
\label{umsetzung_training_architektur}
% ===========================

Ut wisi enim ad minim veniam, quis nostrud exerci tation ullamcorper suscipit lobortis nisl ut aliquip ex ea commodo consequat. Duis autem vel eum iriure dolor in hendrerit in vulputate velit esse molestie consequat, vel illum dolore eu feugiat nulla facilisis at vero eros et accumsan et iusto odio dignissim qui blandit praesent luptatum zzril delenit augue duis dolore te feugait nulla facilisi.


% ===========================
\subsection{Experimente}
\label{umsetzung_training_experimente}
% ===========================

 Stet clita kasd gubergren, no sea takimata sanctus est Lorem ipsum dolor sit amet. Lorem ipsum dolor sit amet, consetetur sadipscing elitr, At accusam aliquyam diam diam dolore dolores duo eirmod eos erat, et nonumy sed tempor et et invidunt justo labore Stet clita ea et gubergren, kasd magna no rebum. sanctus sea sed takimata ut vero voluptua. est Lorem ipsum dolor sit amet. Lorem ipsum dolor sit amet, consetetur

 
 





% ===========================
\chapter{Ergebnis}
\label{ergebnis}
% ===========================

Nam liber tempor cum soluta nobis eleifend option congue nihil imperdiet doming id quod mazim placerat facer possim assum. Lorem ipsum dolor sit amet, consectetuer adipiscing elit, sed diam nonummy nibh euismod tincidunt ut laoreet dolore magna aliquam erat volutpat. Ut wisi enim ad minim veniam, quis nostrud exerci tation ullamcorper suscipit lobortis nisl ut aliquip ex ea commodo consequat.


% ===========================
\section{Synthetische Daten}
\label{ergebnis_synthetisch}
% ===========================

Nam liber tempor cum soluta nobis eleifend option congue nihil imperdiet doming id quod mazim placerat facer possim assum. Lorem ipsum dolor sit amet, consectetuer adipiscing elit, sed diam nonummy nibh euismod tincidunt ut laoreet dolore magna aliquam erat volutpat. Ut wisi enim ad minim veniam, quis nostrud exerci tation ullamcorper suscipit lobortis nisl ut aliquip ex ea commodo consequat.


% ===========================
\section{Reale Daten}
\label{ergebnis_real}
% ===========================

Nam liber tempor cum soluta nobis eleifend option congue nihil imperdiet doming id quod mazim placerat facer possim assum. Lorem ipsum dolor sit amet, consectetuer adipiscing elit, sed diam nonummy nibh euismod tincidunt ut laoreet dolore magna aliquam erat volutpat. Ut wisi enim ad minim veniam, quis nostrud exerci tation ullamcorper suscipit lobortis nisl ut aliquip ex ea commodo consequat.


% ===========================
\chapter{Zusammenfassung}
\label{zusammenfassung}
% ===========================

In dieser Arbeit wurde ein Konzept für die Klassifizierung von Szenarien entwickelt und umgesetzt. Dafür wurden im ersten Schritt beispielhaft sieben Fahrszenarien auf \textit{funktionaler und logischer Ebene} definiert. Auf Basis der \textit{logischen Definitionen} wurden Variablen abgeleitet, die das jeweilige Szenario regelbasiert annotieren können. 

Im zweiten Schritt wurden mit der Simulationssoftware CarMaker 2.160 Kilometer und damit 326.108 Szenen simuliert. Zu jeder Szene wurden Bilddaten und die zuvor abgeleiteten Variablen, sogenannte Signaldaten, erzeugt. Die Bilddaten wurde mit diesen Signaldaten regelbasiert, nach der \textit{logischen Definition}, annotiert und schließlich zu 3-Sekunden-Szenarien zusammengefasst. Im Anschluss wurden einige realen Szenarien manuell annotiert.

Im dritten Schritt wurde für das Training von \acp{KNN} ein Trainingsdatensatz mit fünf Klassen aus 95 Prozent synthetischen und 5 Prozent realen Szenarien erstellt. Insgesamt wurden acht unterschiedliche \ac{KNN}-Architekturen designed, trainiert und getestet. Das beste Modell erreicht Genauigkeiten von 95 Prozent bei der Klassifizierung von realen Testszenarien.

% ===========================
\section{Ergebnis und Diskussion}
\label{zusammenfassung_ergebnis}
% ===========================

Das Ziel dieser Arbeit war es, ein Proof-of-Concept für die Erkennung von Fahrszenarien auf Basis von überwiegend synthetischen Bilddaten zu erstellen. Der Fokus lag dabei auf den folgenden Punkten:

\begin{itemize}
\item Trainingsdaten müssen nicht mehr aufwendig manuell annotiert werden
\item Szenarien werden auf der Basis von Bilddaten klassifiziert
\item Bisher unbekannte Szenarien können gefunden werden
\end{itemize}

Im zweiten Schritt des Ansatzes dieser Arbeit wurde eine Methodik entwickelt und umgesetzt, mit der synthetische Bilddaten simuliert und annotiert werden können. Damit ist es in Zukunft möglich große Datenmengen für das Training von neuronalen Netzen für die Erkennung von Fahrszenarien zu generieren. Diese Methodik ist nicht auf bestimmte Szenarien festgelegt und kann in nachfolgenden Arbeiten und in der Praxis für weitere Fahrszenarien angewandt werden. Im dritten Schritt dieser Arbeit wurde zusätzlich gezeigt, dass sehr gute Klassifizierungsgenauigkeiten mit 95 Prozent synthetischen und nur 5 Prozent realen Trainingsdaten erreicht werden können. Damit ist es möglich den manuellen Aufwand für die Annotation von realen Trainingsdaten stark zu reduzieren.

In vorherigen Arbeiten, wie in Abschnitt \ref{grundlagen_fahren_szenarien} beschrieben, wurden Fahrszenarien auf der Basis von Signaldaten klassifiziert. Nach bestem Wissen des Autors wurden in dieser Arbeit zum ersten Mal Fahrszenarien mit ausschließlich Bilddaten klassifiziert. Damit soll die Grundlage gelegt werden, um in weiteren Arbeiten zusätzliche Fahrszenarien klassifizieren zu können und schließlich bisher unbekannte Szenarien für die Absicherung von hochautomatisierten \ac{FAS} zu finden. Die Grundidee ist es, dass ein Klassifikator, der mit allen bisher bekannten Szenarien trainiert wurde, bekannte Szenarien klassifizieren und unbekannte Szenarien identifizieren kann. Dieses Vorgehen wurde in dieser Arbeit mit fünf beispielhaften Szenarien in einem Proof-of-Concept gezeigt.
 
 % ===========================
 \section{Ausblick}
 \label{zusammenfassung_ausblick}
 % ===========================

Der nächste Schritt ist, wie in Abschnitt \ref{zusammenfassung_ergebnis} beschrieben, die Übertragung dieses Ansatzes auf alle bisher bekannten Szenarien um unbekannte Fahrszenarien zu identifizieren. Damit kann ein Beitrag zur Absicherung von \ac{FAS} geliefert werden.

Eine zweite Möglichkeit für nachfolgende Arbeiten ist es, die \ac{KNN}-Architekturen weiter zu verbessern. In dieser Arbeit wurden oft Standardkonfigurationen von Keras \cite{chollet2015keras} für die einzelnen Schichten verwendet und keine umfangreichen empirischen Untersuchungen durchgeführt. Es ist anzunehmen, dass die Genauigkeit noch gesteigert und die Trainingszeit reduziert werden kann.

Eine zusätzlicher, wichtiger nächster Schritt ist die Vergrößerung der Datenvarianz. Bisher wurden Szenarien ausschließlich auf einer 4-spurigen Autobahn bei Tageslicht simuliert. Für die Anwendung in der Praxis müssen auch Fahrten unter anderen Umständen (bei Nacht, in der Stadt etc.) simuliert und für das Training eines Klassifikators verwendet werden. Nur so kann die Anwendbarkeit in der Praxis sichergestellt werden.

Eine vierte Möglichkeit ist die Erweiterung des Klassifikators. Wie zuvor beschrieben, fokussierten sich bisherige Arbeiten auf die Klassifizierung auf Basis von Signaldaten. In dieser Arbeit werden ausschließlich Bilddaten verwendet. Ein möglicher nächster Schritt ist die Kombination beider Ansätze. Es könnten \ac{KNN}-Architekturen entwickelt werden, die sowohl Bilddaten als auch Signaldaten für die Klassifizierung verwenden. Dadurch könnten möglicherweise einzelne Klassen, die bisher weniger gut erkannt werden, mit einer höheren Genauigkeit klassifiziert werden.

In dieser Arbeit wurde ein Ansatz für die Klassifizierung von Fahrszenarien auf Basis von simulierten Bilddaten entwickelt und umgesetzt. Dieser Ansatz kann als Grundlage für verschiedene weitere Untersuchungen herangezogen werden, um in Zukunft autonome Fahrfunktionen abzusichern.







\cleardoublepage
\printbibliography
\renewcommand{\bibname}{Literaturverzeichnis}
\addcontentsline{toc}{chapter}{Literaturverzeichnis}

\cleardoublepage
\appendix

% ===========================
\chapter*{Appendix}
\label{appendix}
% ===========================

\begin{flushleft}

% ===========================
\section*{Generierte Daten dieser Arbeit}
\label{appendix_data}
% ===========================

Unter dem nachfolgenden Link sind alle generierten Daten dieser Arbeit verfügbar:

\vspace{0,5cm}
\url{https://bit.ly/2A6hMKH}
\vspace{0,5cm}

Im Ordner \textit{Raw Data CarMaker} sind die generierten Signal- und Bilddaten aus der Simulationssoftware CarMaker verfügbar. Im Ordner \textit{Processed Data} sind alle synthetischen und realen annotierten Szenarien als Numpy-Arrays gespeichert. Die trainierten \acp{KNN}-Modelle und alle dazugehörigen Ergebnisdaten sind im Ordner \textit{Trained Neural Networks} zu finden.


% ===========================
\section*{Python-Code dieser Arbeit}
\label{appendix_code}
% ===========================

Unter dem nachfolgenden Link ist der gesamte Python-Code, der in dieser Arbeit geschrieben wurde, zu finden:

\vspace{0,5cm}
\url{https://github.com/manuelhk/master-thesis-python}
\vspace{0,5cm}

Der gesamte Code ist in der Datei \textit{README.md} beschrieben.

\end{flushleft}

\cleardoublepage

\end{document}
